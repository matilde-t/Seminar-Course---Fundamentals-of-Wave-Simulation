\documentclass{scrartcl}
\usepackage[utf8]{inputenc}

\usepackage{todonotes}
\title{Review: \\Approximate Riemann solvers, HLL and its Variants}
\author{Anonymous reviewer} % DO NOT UNDER ANY CIRCUMSTANCES CHANGE THIS LINE
\date{January 2024}

\begin{document}

\maketitle

\section{Summary of paper}
Give a short (roughly 1/4 page) summary of your understanding of the paper.

\section{Review}

The paper is very well written: I easily understood most of the content on the first read. I only have a couple of minor comments.

In general, given that the pictures count into the page limit and there are still no numerical results, some paragraphs can be summarized a bit more. However, I am not sure about how strict the page limit is.

The \textit{Abstract} title can be removed if the section is not used, same goes for \textit{Conclusion}.

\subsection{Introduction}

$\mathcal{Q}$ and $\mathcal{F}$ in (2) can be explained in more detail, especially their time and space relations (write them explicitly at least once).

The first paragraph on the second page can be summarized: the target audience is probably familiar with the concept of discretization. The derivation of the method in (5) can be hinted (it comes from Euler at a first glance). The meaning of the inter cell numerical flux in (6) is also not very clear.

\textsc{Typos}: whith, Riemman, imposible.

\subsection{The Shallow Water Equations}

It's better to state at least once, in (8), the time and space dependencies.

\textsc{Typos}: asociated.

\subsection{HLL Solver}

It would be useful to have a quick refresh on what the Rankine-Hugoniot conditions are, given that they are used multiple times. 

The difference between HLL and HLLE is not very clear. It seems that HLL is not complete on its own and needs another assumption to complete, but maybe it is worth emphasising. 

\textsc{Typos}: sepeds, avergae, togheter, execively.

\subsection{The HLLC Solver}

This section was really well made. It may be nice explaining why this solver is called HLL\textbf{C}.

\textsc{Typos}: expressionsthat.

\subsection{The HLLEM Solver}

In (32) there is $\delta_*$, but underneath it says "$\delta_*(\tilde{\mathcal{Q}})$ is a diagonal matrix". It is unclear whether this value is a diagonal matrix or a scalar.

\textsc{Typos}: authrs, strenght, unfortunatedly.


\end{document}


