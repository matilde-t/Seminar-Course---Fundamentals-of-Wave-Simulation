\documentclass{beamer}

\newcommand{\ca}{\mathcal{A}}
\newcommand{\cb}{\mathcal{B}}
\renewcommand{\d}{\Delta}

\usepackage{concmath} 
\usetheme{Boadilla}
\usepackage[utf8]{inputenc}
\usepackage{helvet}
\usepackage[english]{babel}

\usepackage{natbib} % for the bibliography
\setcitestyle{numbers}

\usecolortheme{tum}
\useoutertheme{tum}

\setbeamerfont{author}{size=\footnotesize}
\setbeamerfont{date}{size=\scriptsize}
\setbeamerfont{date}{size=\scriptsize}

\useinnertheme{rectangles}

\usepackage{pgf}  
\usepackage{tikz}
\logo{\pgfputat{\pgfxy(-0.2, 8.5)}{\pgfbox[right,top]{
\begin{tikzpicture}[y=0.38pt, x=0.38pt,yscale=-1, inner sep=0pt, outer sep=0pt]
\begin{scope}[cm={{1.25,0.0,0.0,-1.25,(0.0,35.4325)}}]
    \path[fill=tum,nonzero rule] (4.8090,23.2950) -- (4.8090,-0.0020) --
      (9.8590,-0.0020) -- (9.8590,23.2600) -- (15.4730,23.2600) -- (15.4730,-0.0020)
      -- (31.5390,-0.0020) -- (31.5390,23.0140) -- (37.2580,23.0140) --
      (37.2580,0.0060) -- (42.5550,0.0060) -- (42.5550,23.0140) -- (48.3440,23.0140)
      -- (48.3440,0.0060) -- (53.6410,0.0060) -- (53.6410,28.3460) --
      (26.4530,28.3460) -- (26.4530,5.1580) -- (20.6290,5.1580) -- (20.6290,28.3110)
      -- (-0.0000,28.3110) -- (-0.0000,23.2950) -- (4.8090,23.2950) -- cycle;
\end{scope}
\end{tikzpicture}
}}}

\setbeamertemplate{title page}
{
	\vbox{}
	\vfill
	\begin{flushleft}
		\begin{beamercolorbox}[sep=8pt,left]{title}
			\usebeamerfont{title}\inserttitle\par%
			\ifx\insertsubtitle\@empty%
			\else%
				\vskip0.25em%
				{\usebeamerfont{subtitle}\usebeamercolor[fg]{subtitle}\insertsubtitle\par}%
			\fi%
    	\end{beamercolorbox}%
    	\vskip1em\par
		\begin{beamercolorbox}[sep=8pt,left]{author}
		\usebeamerfont{author}\insertauthor
		\end{beamercolorbox}
		\begin{beamercolorbox}[sep=8pt,left]{institute}
		\usebeamerfont{institute}\insertinstitute
		\end{beamercolorbox}
		\begin{beamercolorbox}[sep=8pt,left]{date}
		\usebeamerfont{date}\insertdate
		\end{beamercolorbox}\vskip0.5em
		{\usebeamercolor[fg]{titlegraphic}\inserttitlegraphic\par}
	\end{flushleft}
	\vfill
}

\mode<presentation>

\title{Source Terms}
% \subtitle{Source Terms}

\author{Matilde Tozzi}
\institute[]{Seminar Course - Fundamentals of Wave Simulation - Solving Hyperbolic Systems of PDEs}
\date[January 2024]{January 2024}

\renewcommand{\emph}[1]{\textcolor{tum}{\textbf{#1}}}

\begin{document}
\beamertemplatenavigationsymbolsempty

\begin{frame}
	\titlepage
\end{frame}

% 2. Slide: TOC
\begin{frame}
	\frametitle{Table of contents}
	\tableofcontents
\end{frame}


%%%%%%%%%%%%%%%%%%% From Conservation Laws to Balance Laws
\section{From Conservation Laws to Balance Laws}
\begin{frame}
	\frametitle{From Conservation Laws to Balance Laws}
	Our reference equation is
	\begin{equation}
		q_t +f(q)_x = \psi(q)
	\end{equation}
	where
	\begin{itemize}
		\item the homogeneous equation $q_t +f(q)_x = 0$ is \emph{hyperbolic}
		\item $\psi(q)$ (the \emph{source terms}) don't depend on derivatives of $q$
		      \begin{itemize}
			      \item $\Rightarrow$ $q_t = \psi(q)$ is an independent system of ODEs
		      \end{itemize}
	\end{itemize}
\end{frame}

%%%%%%%%%%%%%%%%%%% Godunov-Strang splitting
\section{Godunov-Strang splitting}

%%%%%%%%%%%%%%%%%%% The Advection-Reaction Equation
\subsection{The Advection-Reaction Equation}

\begin{frame}
	\frametitle{The Advection-Reaction Equation}

	A standard example that will be used to illustrate the following numerical methods is the \emph{advection-reaction equation}

	\begin{equation}\label{eq:advec}
		q_t + \bar{u}q_x=-\beta q.
	\end{equation}

	It can be seen as the model for the transport along a flow of a radioactive substance, where

	\begin{itemize}
		\item $\beta$ is the \emph{decay rate}
		\item $\bar{u}$ is the (constant) \emph{transport speed}
		\item $q(x,0)= \mathring{q}(x)$ is the \emph{initial condition}.
	\end{itemize}
	\pause
	\begin{block}{Exact solution}
		Along the characteristic $\frac{dx}{dt}=\bar{u}$ we have $\frac{dq}{dt}=-\beta q$ and it follows that

		\begin{equation}\label{eq:advec_sol}
			q(x,t) = e^{-\beta t}\mathring{q}(x-\bar{u}t).
		\end{equation}
	\end{block}

\end{frame}

\begin{frame}
	\frametitle{The Advection-Reaction Equation: Plot}
	\begin{figure}[!ht]
		\centering
		\includegraphics[width=.7\textwidth]{../Advection.png}

		\caption{Evolution of the exact solution of the advection-reaction equation with $\bar{u}=1$, $\beta=1$, and $\mathring{q}=\text{Gaussian}(0.25,0.003)$.}
		\label{fig:exact}
	\end{figure}
\end{frame}

%%%%%%%%%%%%%%%%%%% The Unsplit Method
\subsection{The Unsplit Method}

\begin{frame}
	\frametitle{The Unsplit Method}
	Dor this specific example we can easily compute an \emph{unsplit method}

	\begin{align*}\label{eq:unsplit}
		q_t                          & = -\bar{u}q_x-\beta q                                              \\
		\frac{Q^{n+1}_i-Q^n_i}{\d t} & = -\bar{u} \frac{Q^n_i-Q^n_{i-1}}{\d x}-\beta Q^n_i                \\
		Q^{n+1}_i                    & = Q^n_i-\bar{u} \frac{\d t}{\d x}(Q^n_i-Q^n_{i-1})-\d t\beta Q^n_i
	\end{align*}

	which is first-order accurate and stable for $0<\bar{u}\frac{\d t}{\d x}\leq1$.
\end{frame}

\begin{frame}
	\frametitle{Taylor Expansion of the Exact Solution}
	\begin{block}{Note}
		The full Taylor expansion of \eqref{eq:advec} can be written formally as

		\begin{equation}\label{eq:sol_op}
			\begin{gathered}
				e^{-\d t(\bar{u}\partial_x+\beta)}q(x,t):=q(x,t+\d t)=\\
				=\sum_{j=0}^{\infty}\frac{(\d t)^j}{j!}\partial_t^jq(x,t)=\sum_{j=0}^{\infty}\frac{(\d t)^j}{j!}(-\bar{u}\partial_x-\beta)^jq(x,t).
			\end{gathered}
		\end{equation}

		The operator $e^{-\d t(\bar{u}\partial_x+\beta)}$ is called \emph{solution operator} for the equation \eqref{eq:advec} over a time step of length $\d t$.
	\end{block}
\end{frame}












%%%%%%%%%%%%%%%%%%% Godunov Splitting
\subsection{Godunov Splitting}

\begin{frame}
	\frametitle{Godunov Splitting}
	In the case of the advection equation, we can split it into two subproblems:

	\begin{equation}\label{eq:probA}
		\text{Problem A: } q_t+\bar{u}q_x=0,
	\end{equation}

	\begin{equation}\label{eq:probB}
		\text{Problem B: } q_t = -\beta q.
	\end{equation}

	The idea is to apply the two methods in an alternating manner, using standard solving stategies, e.g.:

	\begin{equation}\label{eq:stepA}
		\text{A-step: } Q_i^* = Q_i^n - \frac{\bar{u}\d t}{\d x} (Q_i^n-Q_{i-1}^n),
	\end{equation}

	\begin{equation}\label{eq:stepB}
		\text{B-step: } Q_i^{n+1} = Q_i^*-\beta\d tQ_i^*.
	\end{equation}

\end{frame}

\begin{frame}
	\frametitle{Unsplit Method vs Godunov Splitting}

	One may think that given that both $Q_i^*$ and $Q_i^{n+1}$ are calculated using $\d t$, the solution is valid for time $2\d t$, but it is not really the case: in fact if we combine the two stages and eliminate $Q_i^*$, we obtain

	\begin{equation*}
		Q_i^{n+1} = Q_i^n -\frac{\bar{u}\d t}{\d x}(Q_i^n-Q_{i-1}^n)-\beta\d tQ_i^n +\frac{\bar{u}\beta\d t^2}{\d x}(Q_i^n-Q_{i-1}^n),
	\end{equation*}

	which differs from the unsplit method for the last term:

	\begin{equation*}
		Q^{n+1}_i = Q^n_i-\bar{u} \frac{\d t}{\d x}(Q^n_i-Q^n_{i-1})-\d t\beta Q^n_i
	\end{equation*}

\end{frame}










%%%%%%%%%%%%%%%%%%% General Formulation
\subsection{General Formulation}

















%%%%%%%%%%%%%%%%%%% Strang Splitting
\subsection{Strang Splitting}





















%%%%%%%%%%%%%%%%%%% Accuracy
\subsection{Accuracy}
















%%%%%%%%%%%%%%%%%%% Implicit Methods and Choice of ODE Solver
\section{Implicit Methods and Choice of ODE Solver}















%%%%%%%%%%%%%%%%%%% Stiff and Singular Source Terms and the Associated Numerical Difficulties
\section{Stiff and Singular Source Terms and the Associated Numerical Difficulties}




















%%%%%%%%%%%%%%%%%%% Bibliography
\begin{frame}
	\frametitle{Bibliography}
	\bibliographystyle{amsalpha}
	\begin{thebibliography}{1}

		\bibitem{leveque}
		R.~J.~LeVeque, \emph{Finite Volume Methods for Hyperbolic Problems}.\hskip 1em plus
		0.5em minus 0.4em\relax Cambridge: Cambridge University Press, 2002.

		\bibitem{github}
		\url{https://github.com/matilde-t/SeminarCourse-FundamentalsOfWaveSimulation}

		\bibitem{clawpack}
		\url{https://github.com/clawpack/apps/tree/master/fvmbook/chap17}

		\bibitem{riemann}
		\url{https://github.com/clawpack/riemann_book}

	\end{thebibliography}
\end{frame}

\end{document}